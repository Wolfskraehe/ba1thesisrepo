\chapter{Related work}
\label{chap:rlw}
M.C.Riff and E.Montero proposed a parameter free CLONALG variant in \cite{RIFF09}. They tested different strategies involving the elemination of the population size, the selection size or both at the same time. The idea behind the modification was the use of a reinforcement learning model as mentioned in chapter \ref{chap:ais}. According to their results making the selecetion size dynamic alone yielded the best increase in performance when solving a TSP, which is comparable to the results of the ESPC in chapter \ref{chap:eva}. The ESPC achieved the same result with a different strategy.\\
For the ESPC variant in this thesis some techniques from S.M.Garret's work on parameter free clonal selection desribed in \cite{Garret04} were used. Garret applied different parameter free variants of the CLONALG to different multimodal problems. The modification was based on the evolution strategy where a static parameter is used to alter the dynamic parameters. Garret was able to successful eliminate all parameters except population size which still has to be initialized with a value best suited for the specific task. The adaptive clonal selection ACS was also tested with different strategies eliminating one parameter at the time. The results showed that the ACS with all parameters eleminated could outperfom the compared algorithms. The author also speculates that the ACS could be especially well suited for dynmamic multimodal problems, where the optima are changing over time. This could not be tested in his work \cite{Garret04}.\\
The designer of the original CLONALG L.E.de Castro and F.J.Von Zuben also proposed some parameter control based on sensitivity analysis \cite{DEC02}. The tuned parameters were tested on a 30 nodes TSP and achieved better results than the untuned algorithm. This parameters were also applied to the evaluation in chapter \ref{chap:eva} but yielded much different results than in de Castro and Von Zuben's work.\\
L.Pasti and L.N.de Castro proposed a neuro-immune hybrid algorithm in \cite{Pasti06}. A single layer self organized neural network was combined with the clonal selection from the immune system. The results showed that this hybrid was able to produce high quality results for many TSP and had good performance in solving more difficult TSP with a high amount of nodes.\\
Wei-Dong Sun et al tested an artificial immune system based on the immune network theory on small TSP to a range of 100 nodes. Their proposed algorithm could achieve the best score for the TSP in 63 percent of runs as stated in \cite{sun}. 