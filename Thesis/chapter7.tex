\chapter{Conclusion}
\label{chap:con}
When using the 10000 evaluations without improvement as a stopping criteria, the algorithms based on the CLONALG show a worse performance compared to the greedy algorithm on average. The results still show potential for these algorithms based on their better performance with smaller TSP espcecially in finding the shorter route or in case of one TSP in finding the best route. The clonal based algorithms also terminated faster which can be an advantage for certain applications even if the overall quality of the result is worse.\\
Static tuning of the parameters made the performance of the CLONALG worse. The dynamic tuning of the selection size based on evolution strategy however produced a better performance than the original CLONALG. This shows that adaptive tuning is very promising for the cloning selection algorithms as also seen in the related work of \cite{Garret04} and \Cite{RIFF09}.\\
A different stopping criteria of 25 seconds and only one run on every TSP yielded better results for the CLONALG variants. All variants achieved a better performance and could produce more shorter routes compared to the Greedy algorithm than with the first stopping criteria. All tested algorithms are very basic and not optimized for solving the TSP but still achieved a good performance on small TSP in a range under 100 nodes. Adapting the cloning parameters during runtime and giving the algorithm enough time are beneficial to the performance.