%----------------------------------------------------------------
%
%  File    :  chapter2.tex
%
%  Authors :  Keith Andrews, IICM, TU Graz, Austria
%             Manuel Koschuch, FH Campus Wien, Austria
% 
%  Created :  22 Feb 96
% 
%  Changed :  30 Oct 2008
% 
%----------------------------------------------------------------


\chapter{Biological Immune System}
\label{chap:bis}

A biological immune system (BIS) has many features that are useful in machine learning. It can be described with the following terms [TAN16]:
\begin{itemize}
	\item 	Distributed
	\item 	Parallel
	\item 	Multi-level
	\item 	Distinguishes between self and non-self
	\item 	Noise resistant
	\item 	Self-organized
	\item 	Associative memory
	
\end{itemize}\\

It is distributed because there is no need for a central control instance, the BIS acts where it need to and it does so immediately without supervising. It can operate on multiple parts of the body simultaneously thus it is parallel. The BIS works in different levels. At first there is the physical barrier, the skin and the body fluids. If this level fails the innate immune system responds, which is a pre-programmed immune reaction that can respond to known and non-changing threats. Finally, there is the adaptive immune system if all of the previous levels fail. This is the system most interesting for computational science, because of its ability to learn on its own. A very important ability of the BIS is the distinguishing between any self-element of the body and the non-self or potential threat. It is noise resistant because it can react to variations of known threats. The BIS does not need any supervising through the brain or any other central system to organize its work flow, therefore its self-organized. Finally, it has an associative memory which is used to react on similar and variant of threats already encountered.
As mentioned above, the adaptive immune system is the most interesting part of the BIS for modelling an artificial immune system (AIS). This system uses many principles to be effective, in terms of computational science the principles of negative selection and clonal selection are specifically interesting.

\section{Negative Selection}

The adaptive immune system uses two kind of lymphocytes to counter a threat. The T-Lymhocyte and the B-Lymphocyte. Both have different roles but both lymphocytes must have the ability to distinguish between the self and the non-self-cells in the body. A fault in this system can not only lead to an infection, it could trigger an autoimmune reaction because a self-cell could be identified as a non-self threat. To avoid this the Lymphocytes are generated through the process of negative selection. While the B-Cells will be developed in the bone marrow and the T-Cells in the thymus the process is exactly the same. Both cell types are presented to a wide range of self-cells. If any of them react to such a self-cell, the Lymphocyte will be killed and another will be generated. This process is repeated till there are only B- and T-Cell which react to non-self cells [TAN16]. This process will be imitated in the AIS in generating detector sets.
\\\\
After that the lymphocytes will be released into the tissue and the blood system. The B-cells are called antibodies (Ab) in this context, and any non-self cell is called an antigen (Ag). The B-cell has the ability to recognize and dock to a specific Ag. The ability to recognize an Ag is called affinity. If an Ab has a high affinity to an Ag it is especially good in recognizing and countering this Ag. Because the generation of Lymphocytes and therefore the affinity to different Ag’s is random, additional measures are necessary to improve effectiveness.

\section{Clonal Selection}

If a B-Cell encounters an antibody it is able to proliferate (divide) into multiple terminal cells which are clones of the cell. The cell is not only cloned, the different clones will be mutated to improve the affinity to the antigen. The process of cloning and mutating will be repeated till a population and affinity threshold is reached which ensures the most efficient response to the Ag. Only the Ab’s with the highest affinity score will be cloned and mutated. If the affinity is high enough, the B-Cell can proliferate into memory cells which will stay after the response and ensure that a secondary response to a similar Ag will be much faster than the initial one [DEC02]. This memory cells represent the associative memory of the adaptive immune system. 
\\\\
The principles of negative selection and clonal selection are important concepts in designing an artificial immune system. The clonal selection aspect of the BIS is basically the learning system. The cloning process is a form of reinforcement learning and leads to a continues improvement [DEC02].
The mutation process itself is called affinity maturation. Random changes in the genes leads to changes in affinity in every single clone. The mutation process is invers to the affinity level. Higher affinity level means lower mutation rate [DEC02].  
Clonal selection is based on the basic evolutionary theory of Charles Darwin.\\\\\\ The three basic principles are [DEC02]:
\\
\begin{itemize}
	\item 	repertoire diversity (high population of Ab’s)
	\item 	genetic variation (random changes to the population (blind variation))
	\item 	natural selection (high affinity Ab’s will reproduce and maintained) 
	
\end{itemize}

These are high level abstract concepts and are only used for a very brief overview of the immune system. The BIS is far more complex but the details are out of scope of this thesis.



