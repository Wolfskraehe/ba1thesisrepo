%----------------------------------------------------------------
%
%  File    :  chapter2.tex
%
%  Authors :  Keith Andrews, IICM, TU Graz, Austria
%             Manuel Koschuch, FH Campus Wien, Austria
% 
%  Created :  22 Feb 96
% 
%  Changed :  30 Oct 2008
% 
%----------------------------------------------------------------


\chapter{Biological Immune System}
\label{chap:bis}

\nocite{DEC02}
\nocite{NAN08}
\nocite{PAM17}
\nocite{RIFF09}

A biological immune system (BIS) has many features useful for machine learning. It can be described with the following terms \cite{tan2016artificial}:
\begin{itemize}
	\item 	Distributed
	\item 	Parallel
	\item 	Multi-level
	\item 	Distinction between self and non-self
	\item 	Noise resistant
	\item 	Self-organized
	\item 	Associative memory
	
\end{itemize}
Distribution means there is no need for a central control instance, the BIS acts where it needs to and it does so immediately without supervision. It is able to operate on multiple parts of the body simultaneously, thus it is parallel. The BIS works on different levels: first there is the physical barrier, the skin and body fluids. If this level fails the innate immune system responds, a pre-programmed immune reaction able to respond to known and unchanging threats. The last level consists of the adaptive immune system, which takes over if all of the previous levels fail. This system holds the most interest for computational science because of its ability to learn on its own. An important ability of the BIS is the distinction between any self-element of the body and the non-self or potential threat. The BIS is noise resistant because it can react to variations of known threats. The BIS does not need any supervising through the brain or any other central system to organize its work flow, therefore its self-organized. Finally, the associative memory is used to react to similar and variable threats already encountered.\\
As mentioned above, the adaptive immune system is the most interesting part of the BIS for modelling an artificial immune system (AIS). The BIS uses many principles to be effective, in terms of computational science the principles of negative selection and clonal selection are especially interesting.

\section{Negative Selection}

The adaptive immune system uses two kinds of lymphocytes to counter a threat, the T lymphocyte and the B lymphocyte. A lymphocyte is one variation of a white blood cell \cite{immunebio}.  Both lymphocytes have different roles but both need the ability to distinguish between self and non-self cells in the body. A fault in this system can not only lead to an infection, it could also trigger an autoimmune reaction where a self-cell is wrongly identified as a threat. To avoid these mistakes the lymphocytes are generated through a process of negative selection. Although B cells are developed in bone marrow and T cells in the thymus the process is exactly the same. Both cell types are presented with a wide range of self-cells. If any of the lymphocytes react to a self-cell, the lymphocyte is killed and another one generated. This process is repeated till there are only B- and T-Cells which react to non-self cells \cite{tan2016artificial}. This process is imitated in the AIS in generating detector sets.
\\\\
After the negative selection the lymphocytes are be released into the tissue and blood system. In this context B cells are called antibodies (Ab), and any non-self cell is called an antigen (Ag). B cells have the ability to recognize and dock to a specific Ag. The ability to recognize an Ag is called affinity. Ab with a high affinity to an Ag are especially good at recognizing and countering this Ag. Because the production of lymphocytes and therefore the affinity to different Ag is random, additional measures are necessary to improve effectiveness.

\section{Clonal Selection}

If a B-Cell encounters an Ab it is able to proliferate (divide) into multiple terminal cells which are clones of the cell. In addition the different clones are mutated to improve affinity to the Ag. The process of cloning and mutating is repeated until a population and affinity threshold is reached, ensuring the most efficient response to the Ag. Only Ab with the highest affinity scores are cloned and mutated. If affinity is high enough, the B cell can proliferate into memory cells which will stay after the initial immune response and ensure that a secondary response to a similar Ag will be much faster than the initial one \cite{DEC02}. These memory cells represent the associative memory of the adaptive immune system. 
\\\\
 The principles of negative selection and clonal selection are important concepts in designing an AIS.The clonal selection aspect of the BIS is basically a learning system. The cloning process is a form of reinforcement learning and leads to a continued improvement \cite{DEC02}.
The mutation process itself is called affinity maturation. Random changes in the genes lead to changes in affinity in every single clone. The mutation process is inverse to the affinity level,  a higher affinity level means a lower mutation rate \cite{DEC02}.  
Clonal selection is reflected in the basic evolutionary theory of Charles Darwin.\\\\\\ The three basic principles are \cite{DEC02}:
\\
\begin{itemize}
	\item 	repertoire diversity (high population of Ab's)
	\item 	genetic variation (random changes to the population [blind variation])
	\item 	natural selection (high affinity Ab's will reproduce and be maintained) 
	
\end{itemize}

The mentioned mechanisms of the BIS are high level abstract concepts and are only used for a very brief overview of the immune system. The BIS is far more complex but the details are out of scope of this thesis.



