%----------------------------------------------------------------
%
%  File    :  abstracts.tex
%
%  Authors :  Keith Andrews, IICM, TU Graz, Austria
%             Manuel Koschuch, FH Campus Wien, Austria
% 
%  Created :  22 Feb 96
% 
%  Changed :  30 Oct 2008
% 
%----------------------------------------------------------------


% --- English Abstract ----------------------------------------------------

\cleardoublepage

%\selectlanguage{english}

\begin{center}
{\Large\bfseries Abstract}
\end{center}
Artificial immune systems are, along with neural nets and genetic algorithms, an important part of the bio-inspired approaches in machine learning. This thesis examines artificial immune systems and their performance in solving the traveling salesman problem (TSP). The TSP has many applications in industry and in scientific research like drilling of printed circuit boards, computer wiring, order picking in warehouses, vehicle routing and DNA sequencing, therefore new and more efficient ways for finding TSP solutions are always relevant. The CLONALG algorithm is compared to a more conventional greedy algorithm. To achieve this, the Optimization Algorithm Toolkit will be used and the CLONALG, a Greedy algorithm, a tuned CLONALG and a modified CLONALG with parameter control will be compared in solving 17 different TSP. Comparison shows that the clonal algorithms can achieve better results under certain circumstances but are, in their general form, not as efficient as the greedy algorithm. The CLONALG variants show better performance when applied to smaller TSP in the range from 22 to 100 nodes and can also find the optimal route for one TSP contrary to the Greedy algorithm. The CLONALG and most clonal selection algorithms work with static parameters. This thesis examines if parameter control is beneficial for solving the traveling salesman problem and shows that dynamic adaptation of certain parameters during runtime can enhance performance of the algorithm. A variant of the CLONALG with dynamic selection size (how many members of the initial population will be cloned) is implemented, tested and shows better performance than the original algorithm and the tuned one. The modified CLONALG is able to find shorter routes for many of the presented TSP compared to the original algorithm.

%\selectlanguage{austrian}
