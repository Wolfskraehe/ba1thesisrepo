%----------------------------------------------------------------
%
%  File    :  chapter1.tex
%
%  Authors :  Keith Andrews, IICM, TU Graz, Austria
%             Manuel Koschuch, FH Campus Wien, Austria
% 
%  Created :  22 Feb 96
% 
%  Changed :  24 March 2009
% 
%----------------------------------------------------------------


\chapter{Introduction}
\label{chap:intro}

Bio-inspired approaches in machine learning mimic biological functions and concepts like the information exchange of neurons in the human brain or the evolution of genes through selection. The biological system often has very efficient methods for learning and adaptation therefore the bio-inspired approaches were very successful over the past years in the field of machine learning. The immune system got more attention in the computational intelligence field recently over the last 20 years \cite{tan2016artificial}.\\\\
The biological immune system has a number of desireable traits for learning in computational science as mentioned in Chapter \ref{chap:bis}. These traits can be used in algorithms to build an artificial immune system for solving a wide range of problems as shown in Chapter \ref{chap:ais}. The artificial immune system can be applied to similar fields like neural nets or genetic algorithms but are also efficient in combination with those algorithms \cite{Pasti06}. One of these problem fields is the traveling salesman problem. The artificial immune system was successfully applied to the TSP by \cite{DEC02} and \cite{sun}. In this thesis the overall performance of the AIS in solving the TSP will be examined especially compared to a more conventional method in solving them. The artificial immune system works with different approaches taken from the biological immune system. One of these approaches is the clonal selection which was formalized into an algorithm by \cite{DEC02}. In Chapter \ref{chap:eva} this algorithm called CLONALG will be applied to a set of TSP.\\
The CLONALG is originally a static algorithm with fixed parameters. The work of \cite{RIFF09} and \cite{Garret04} showed that in dynamically changing these parameters, the performance of the algorithm can be enhanced. In Chapter \ref{chap:eva} a tuned CLONALG and a variant with dynamic parameter control will be applied to the same set of TSP and evaluated.\\\\
The goal is to examine if the AIS is able to achieve good results in solving the TSP and especially under which circumstances and what parameters within a clonal selection algorithm are responsible for a good performance.



